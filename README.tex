\documentclass{article}


\title{Group Eight Binary Tree Implemented in C University of Montana Spring 2019 CSCI 205}
\date{15-03-2019}
\author{Cole Brooks, Conner Copeland, Brooke Kern, Brendan Hagan}

\begin{document}

\maketitle
\pagenumbering{gobble}
\newpage
\pagenumbering{gobble}
\newpage
\pagenumbering{gobble}

\section{Overview}
\paragraph{}
This is a binary search tree implemented in C by Group 8 in the Spring 2019 section of CSCI 205 at the University of Montana. It supports reading space separated integers from a file specified at runtime through a command line argument. A full list and description of the available functions can be found in the "Function Documentation" section below. In addition to the functions supplied within the source code, the repository includes a makefile, and a bash script for testing the functionality of the program.

\section{Function Documentation}
\begin{itemize}
	\item void initialize(binary\textunderscore tree* bt);
		\begin{itemize}
			\item Sets the size of the new tree to zero.
		\end{itemize}
	\item void insert(binary\textunderscore tree* bt, int item);
		\begin{itemize}
			\item Inserts a new node containing the given item into the tree.
		\end{itemize}
	\item bool search(binary\textunderscore tree* bt, int key);
		\begin{itemize}
			\item Returns whether given key is in the tree.
		\end{itemize}
	\item void printinorder(binary\textunderscore tree* bt);
		\begin{itemize}
			\item Prints the tree contents from smallest node to largest.
		\end{itemize}
	\item void printpreorder(binary\textunderscore tree* bt);
		\begin{itemize}
			\item Prints sequence of nodes derived from preorder traversal.
		\end{itemize}
	\item void printpostorder(binary\textunderscore tree* bt);
		\begin{itemize}
			\item Prints sequence of nodes derived from postorder traversal.
		\end{itemize}
	\item int btsize(binary\textunderscore tree* bt);
		\begin{itemize}
			\item Returns number of node in the tree.
		\end{itemize}
	\item int treeheight(binary\textunderscore tree* bt);
		\begin{itemize}
			\item Returns height of the tree.
		\end{itemize}
\end{itemize}


\end{document}
